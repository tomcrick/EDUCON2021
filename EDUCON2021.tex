\documentclass[conference]{IEEEtran}
%\IEEEoverridecommandlockouts
% The preceding line is only needed to identify funding in the first footnote. If that is unneeded, please comment it out.
\usepackage[british]{babel}
\usepackage[noadjust]{cite}
\usepackage{url}
\usepackage[hyphenbreaks]{breakurl}
%\usepackage{amsmath,amssymb,amsfonts}
%\usepackage{algorithmic}
%\usepackage{graphicx}
%\usepackage{tabularx}
\usepackage{booktabs}
\usepackage{multirow}
\usepackage{lscape}
\usepackage{graphicx}
\usepackage[pdftex,colorlinks=true]{hyperref}
\def\UrlBreaks{\do\/\do-}
\def\BibTeX{{\rm B\kern-.05em{\sc i\kern-.025em b}\kern-.08em
    T\kern-.1667em\lower.7ex\hbox{E}\kern-.125emX}}
\begin{document}

\title{The Impact of COVID-19 and ``Emergency Remote Teaching'' on
  International Computer Science Education Practitioners}
%   \\{\footnotesize \textsuperscript{*}Note: Sub-titles are not captured in Xplore and
% should not be used}
% \thanks{Identify applicable funding agency here. If none, delete this.}

\author{\IEEEauthorblockN{Tom Crick\IEEEauthorrefmark{1}, Cathryn
    Knight\IEEEauthorrefmark{1}, Richard
    Watermeyer\IEEEauthorrefmark{2}, Janet Goodall\IEEEauthorrefmark{1}}
\IEEEauthorblockA{\IEEEauthorrefmark{1}School of Education, 
Swansea University, Swansea, UK\\
Email: \{thomas.crick,cathryn.knight,j.s.goodall\}@swansea.ac.uk}
\IEEEauthorblockA{\IEEEauthorrefmark{2}School of Education, 
University of Bristol, Bristol, UK\\
Email: richard.watermeyer@bristol.ac.uk}}

% \author{\IEEEauthorblockN{1\textsuperscript{st} Given Name Surname}
% \IEEEauthorblockA{\textit{dept. name of organization (of Aff.)} \\
% \textit{name of organization (of Aff.)}\\
% City, Country \\
% email address or ORCID}
% % \orcid{0000-0001-5196-9389}
% \and
% \IEEEauthorblockN{2\textsuperscript{nd} Given Name Surname}
% \IEEEauthorblockA{\textit{dept. name of organization (of Aff.)} \\
% \textit{name of organization (of Aff.)}\\
% City, Country \\
% email address or ORCID}
% % \orcid{0000-0002-7574-3090}
% \and
% \IEEEauthorblockN{3\textsuperscript{rd} Given Name Surname}
% \IEEEauthorblockA{\textit{dept. name of organization (of Aff.)} \\
% \textit{name of organization (of Aff.)}\\
% City, Country \\
% email address or ORCID}
% % \orcid{0000-0002-2365-3771}
% \and
% \IEEEauthorblockN{4\textsuperscript{th} Given Name Surname}
% \IEEEauthorblockA{\textit{dept. name of organization (of Aff.)} \\
% \textit{name of organization (of Aff.)}\\
% City, Country \\
% email address or ORCID}
% % \orcid{0000-0002-0172-2035}
% }

\maketitle

\begin{abstract}
The COVID-19 pandemic has imposed ``emergency remote teaching'' across
education globally, leading to the closure of institutions across all
settings, from schools through to universities. This paper looks
specifically at the impact of these disruptive changes to those
teaching the discipline of computer science. Drawing on the
quantitative and qualitative findings from a large-scale international
survey (N=2,483) conducted in the immediate aftermath of closures,
implementation of lockdown measures, and the shift to online delivery
in March 2020, we report how those teaching computer science across
all educational levels (n=327) show significantly more positive
attitudes towards the move to online learning, teaching and assessment
(LT\&A) than those working in other disciplines. When comparing
educational setting, computer science practitioners in schools felt
more prepared and confident than those in higher education; however,
they expressed greater concern around equity and whether students
would be able to access the teaching made available
online. Furthermore, while practitioners across all sectors
consistently noted the potential opportunities of these changes, they
also raised a number of wider concerns on the impact of this shift to
online, especially on workload and job precarity. More specifically
for computer science practitioners, there were concerns raised
regarding the ability to effectively deliver technical topics online,
as well as the impact on various types of formal examinations and
assessment. We also offer informed commentary from this rapid response
international survey on emerging LT\&A strategies that will likely
continue to be constrained by coronavirus into 2021 and possibly
beyond.
\end{abstract}

\begin{IEEEkeywords}
COVID-19, emergency remote teaching, practitioner perceptions,
schools, universities, computer science education
\end{IEEEkeywords}

\section{Introduction}\label{intro}

%\subsection{The Impact of COVID-19}

The impact of the COVID-19 (SARS-CoV-2) global pandemic is currently
incalculable; it has affected, and continues to affect, profound
social suffering, significant cultural disruption, and deep economic
hardship. While indiscriminate in terms of whom it infects, it has
largely punished society’s most vulnerable and less
fortunate~\cite{vonbraun-et-al:2020,lancetcovid:2020,vanlancker+parolin:2020};
worse now, it appears that the virus may have to be tolerated on an
indefinite basis~\cite{kissler-et-al:2020}.

The impact of the pandemic on education systems across the world has
been profound~\cite{unescocovidedu:2020,armitage+nellums:2020},
presenting significant challenges for learning, teaching and
assessment (LT\&A)~\cite{doucet-et-al:2020,oecd:2020,aace:2020} -- and how
face-to-face learning is somehow perceived to be ``higher quality''
than online
approaches~\cite{paechter+maier:ihe2010,scbbcnews:2020}. There have
been significant responses from governments, organisations and
institutions at all levels and settings
internationally~\cite{who:2020}; from major national policy
initiatives to support learners and maintain quality and standards, to
ongoing government inquiries on the longer-term impact of COVID-19 on
education and children’s services.

%\subsection{The Rapid Shift to Digital}

While there has been a rapid shift to ``emergency remote teaching''
during the pandemic, the general impact and efficacy of digital
learning and educational technologies is still unclear in the formal
academic literature, being dependent on specific educational settings
and LT\&A context. Whilst a range of international research studies
have shown benefits of the successful application of digital LT\&A
across a variety of contexts and settings, the widespread adoption,
implementation and evaluation of educational technologies has yet to
be fully
realised~\cite{decodinglearning:2012,means:2014,ecjrc:2017,mayer:2018}.
The research and policy debate regarding the efficacy, utility and
impact of educational technology and digital practice is ongoing, as
exemplified by digital learning and teaching strategies and
initiatives in schools, for example across the
UK~\cite{digscot:2016,dfe:2019} and the USA~\cite{usdoe:2020}, as well
as recent work on digital practice in higher education
(HE)~\cite{jisc:2020,wef:2020}. We have also seen recent evidence
assessments of remote learning~\cite{eefremote:2020}, alongside
guidance on how digital technologies can support
learning~\cite{eefdigtech:2019}.

% wgdcfguidance:2018
% and from the UK's Quality Assurance
% Agency on a taxonomy for digital
% learning~\cite{qaadigtaxonomy:2020}

%\subsection{Computer Science as a Discipline}

% Alongside responses from international professional bodies and learned
% societies~\cite{bcs:2020,acm:2020}

It is clear that the academic discipline of computer science -- and
indeed the wider technology sector -- has much to offer to address the
breadth of societal challenges resulting from the COVID-19 pandemic;
from computational modelling, the use of AI, machine learning and big
data, as well as the wider legal, social, ethical and professional
issues, such as from contact tracing, personal data sharing/storage,
and the use of image recognition and
surveillance~\cite{dwivedi-et-al:ijim2019,ting-et-al:2020,cerf:2020,chun-et-al:2020,rcjbbcnews:2020}. There
has also been recent analysis on the impact of COVID-19 on the
international computer science research community -- as we have seen
across international scientific research communities more
broadly~\cite{oecdcovid19:2020} -- especially on ongoing projects,
careers, and dissemination of work~\cite{msrcovid19:2020}. We build on
recent work looking at the impact on the UK's computer science
education community~\cite{crick-et-al:ukicer2020}; however, there has
been little focus on what this means for international computer
science education and practitioners, especially thinking about the
range of specific disciplinary challenges for LT\&A, across all
settings and levels. This wider work directly links to recent major
international changes to computer science curricula, qualifications
and practice (e.g. in the
UK~\cite{brown-et-al-sigcse2013,brown-et-al-toce2014}), as well as the
emerging focus on the required skills and infrastructure interventions
to support the global post-COVID economic
recovery~\cite{davenport-et-al:educon2020,euparl:2020,mckinsey:2020}.

N.B. With regards to the consistent naming of the discipline through this
paper, we use ``computer science'' to refer to the wider cognate
discipline as represented by computing, ICT, informatics, etc, across
all educational settings.

% other papers:
% brown-et-al-toce2014,davenport-et-al:latice2016,murphy-et-al:programming2017, rscsedproject:2017
% tryfonas+crick:petra2018,crick-et-al:fie2019,davenport-et-al:educon2020

%  (as part
% of a larger survey of the educational workforce (N=2,197), including
% specific work on the UK higher education sector in the
% UK~\cite{watermeyer-et-al:he2020})

\section{Research Questions}\label{rqs}

We thus undertook an anonymous survey of international computer
science educators on their perspectives as practitioners on the rapid shift
to ``emergency remote teaching'' and transitioning online at the
height of the COVID-19 crisis, and what they identify and forecast as
its immediate and prospective impacts. The data was collected in the
immediate aftermath of the forced institutional lockdowns and shift to
online LT\&A. It aimed to provide insight into emerging policy and
practice; impact on practitioners, institutions and thus students; how
might this change the discipline as a result; and what might this mean
for the next academic year and in the longer-term. The analysis and
discussion that follows is based upon the perspectives of n=327
practitioners drawn from across all educational settings,
institutions, and the career hierarchy, and what they recognise to be
the major consequences of COVID-19, the transition to online LT\&A,
and the challenges of maintaining ``continuity of learning''.

The study thus aimed to answer the following questions:

\begin{itemize}
\item How did the perceptions of the rapid move to online LT\&A differ
between practitioners in CS and other disciplines?
\item How did the perceptions of the rapid move online LT\&A differ by
CS practitioner setting?
\item What are the key opportunities and challenges perceived by CS
practitioners globally?
\end{itemize}

% Their accounts document the hopes and fears of the parts of the
% international computer science education community in the face of
% seismic and, as may prove to be, inalterable shifts. The majority of
% respondents tend towards a significantly more positive view of online
% migration than those working in other disciplines, recognising the
% opportunities and potential affordances of the crisis; these
% perceptions were consistent across all educational settings. There
% were some, albeit a minority, who raised a number of generalisable
% concerns on the impact of this shift to online and the challenges
% relating to their roles, their institutions and their sectors as a
% whole.

% \subsubsection*{A Note on Terminology}


\section{Methods}\label{methods}

\subsection{Sample}

% The survey aimed to investigate how the UK computer science education
% workforce has viewed the move to online LT\&A. The target population was
% those who are actively involved in the delivery of LT\&A within the
% education sector. Those who did not meet this criterion were excluded
% from analysis post-hoc.

The survey aimed to investigate how global computer science education
practitioners have viewed the move to online LT\&A. The sample was
taken from a larger survey in which the target population was those
who are actively involved in the delivery of LT\&A within the
education sector. Those who did not meet this criterion were excluded
from analysis post-hoc.

% We adopted a convenience sampling approach in distributing the
% Qualtrics survey whereby a link to the survey was shared via mailing
% lists of professional networks and related education organisations
% (for example, via the Council for Professors and Heads of Computing
% (CPHC), BCS Academy, ACM SIGCSE, the UK's University and College Union
% (UCU), and the British Educational Research Association (BERA) Higher
% Education Special Interest Group, in addition to Twitter and
% LinkedIn). While the use of convenience sampling does not allow
% generalisation to a representative population, this sampling technique
% allowed us to document patterns within the observed population, with
% minimal time and cost restrictions.

We adopted a convenience sampling approach in distributing the
Qualtrics survey whereby a link to the survey was shared via mailing
lists of professional networks and related education organisations
(for example, via ACM SIGCSE, and the UK Council for Professors and
Heads of Computing, in addition to Twitter and LinkedIn). While the
use of convenience sampling does not allow generalisation to a
representative population, this sampling technique allowed us to
document patterns within the observed population, with minimal time
and cost restrictions.

% After excluding those that did not meet the participant requirements
% 2,197 members of the UK education workforce responded to the
% survey. This included 1,148 respondents from the HE (university)
% sector (52.3\%), 279 respondents from FE (12.7\%) and 569 respondents
% from schools (25.9\%). 214 participants indicated that they taught
% computer science. This included 119 from the HE sector (55.6\%), 24
% from FE (11.2\%) and 71 from schools (33.2\%).

After excluding those that did not meet the participant requirement,
2,483 international educational practitioners responded to the
survey. This included 1,465 respondents from the HE (university)
sector (59\%) and 1019 respondents from schools (41\%). 327
participants indicated that they taught computer science. This
included 196 from the HE sector (59.9\%), 131 from schools (40.1\%).

% The survey was launched on 26 March 2020 following the announcement of
% closures across all educational settings in all four nations of the
% UK, and remained open for four weeks. Due to the distribution method
% we cannot calculate the response rate; however, of those who started
% the survey, 84.9\% completed it.

The survey was launched on 26 March 2020 following the announcement of
closures across various educational settings in the UK, Europe and
USA, and closed four weeks later. Due to the distribution method we
cannot calculate the response rate; however, of those who started the
survey, 84.9\% completed it.

\subsection{Questionnaire}

% On the first page of the questionnaire respondents were informed that
% the research was designed to identify their views and experiences of
% the move to online LT\&A in response to COVID-19. The first section of
% the questionnaire consisted of demographic questions in order to
% determine how participant characteristics impacted key variables. In
% order to identify those who are computer scientists, those who
% responded that they worked in the HE sector were asked to select their
% discipline from a list created using the UK Joint Academic Coding of
% Subjects (JACS)
% codes\footnote{\url{https://www.hesa.ac.uk/support/documentation/jacs}}. Those
% who worked in schools and FE were firstly asked if they taught a
% particular subject. Those that responded that they did were then asked
% to select their subject from a list containing all curriculum subjects
% taught across the four nations of the UK.

On the first page of the questionnaire, respondents were informed that
the research was designed to identify their views and experiences of
the move to online LT\&A in response to COVID-19. The first section of
the questionnaire consisted of demographic questions in order to
determine how participant characteristics impacted key variables. In
order to identify those who are computer scientists, those who
responded that they worked in the HE sector were asked to select their
discipline from a list created using the UK Joint Academic Coding of
Subjects (JACS)
codes\footnote{\url{https://www.hesa.ac.uk/support/documentation/jacs}}. Those
who worked in schools were firstly asked if they taught a particular
subject; those that responded that they did were then asked to select
their subject from a list containing all curriculum subjects commonly
taught in schools.

% Demographic questions were followed by Likert and slider-scale
% questions exploring respondents' views of the changes. These included
% questions about how prepared and confident they felt about the move to
% online teaching. In addition, respondents were asked three open-ended questions in
% order to gain their overall insight into the impact of the changes:
% ``{\emph{Please provide any comments of how the online learning and
% teaching changes brought in as a response to COVID-19 will impact
% upon}}'' followed by ``{\emph{your role}}'', ``{\emph{your
% institution}}'' and ``{\emph{your sector of education}}''.

Demographic questions were followed by five-point Likert scale
questions ({\emph{strongly agree; agree; neither agree nor disagree;
disagree; strongly disagree}}) exploring respondents' views of the
changes, about how prepared and confident they felt about the move to
online teaching. In addition, respondents were asked three open-ended
questions in order to gain their overall insight into the impact of
the changes:``{\emph{Please provide any comments of how the online
learning and teaching changes brought in as a response to COVID-19
will impact upon}}'' followed by ``{\emph{your role}}'', ``{\emph{your
institution}}'' and ``{\emph{your sector of education}}''. Ethical
approval for this study was obtained from the institutional research
ethics committee. The survey was piloted on a subsample of the
population before distribution to the wider international computer
science community.

% Ethical approval for this study was obtained from Swansea University's
% College of Arts and Humanities Research Ethics Committee. The survey
% was piloted on a subsample of the population before distribution to
% the wider UK computer science workforce.

\subsection{Analysis}

% Quantitative bivariate chi-square (($\chi^2$)) analysis of the key
% variables was conducted in order to determine overall attitudes to
% online LT\&A and whether there were significant differences between
% those in computer science and those in other disciplines. Chi-square
% tests were utilised due to the categorical nature of the variables and
% to assess whether the observed cell counts are significantly different
% from the expected cell counts. As there were more participants from
% computer science that responded from HE institutions it was necessary
% to control for the effect of setting on these outcomes. Furthermore,
% it could be hypothesised that variables such as gender and years
% working in education may have also predict the participants responses
% to these questions. Therefore, binary logistic regression was used in
% order to control for these variables.

Quantitative bivariate chi-square ($\chi^2$) analysis of the key
variables was conducted in order to determine overall attitudes to
online LT\&A and whether there were significant differences between
those in computer science and those in other disciplines. Furthermore,
comparisons were made between those in schools and HE in order to
determine whether there were significant differences between those
working in different educational settings. Chi-square tests were
utilised due to the categorical nature of the variables and to assess
whether the observed cell counts are significantly different from the
expected cell counts. For the purpose of the chi-square tests, the
Likert scales were coded into a binary `{\emph{agree}}' and
`{\emph{disagree}}' variables. This allowed for ease of interpretation
and to see which groups were significantly more likely than expected
to agree or disagree with each statement.

% Qualitative analysis of the open-ended questions used thematic
% analysis. Thematic analysis has been described as ``a method for
% identifying, analysing and reporting patterns (themes) within
% data''~\cite[p.78]{braun+clarke:2006}. This was done by firstly
% reading through the qualitative responses and numerically coding the
% data to identify whether comments were positive, negative or
% neutral. The responses were coded by two researchers to ensure
% inter-rater reliability (IRR=0.82). Within these codes potential
% themes were identified: ``a theme captures something important about
% the data in relation to the research question and represents some
% level of patterned response or meaning within the data
% set''~\cite[p.82]{braun+clarke:2006}. These themes were reviewed
% rigorously against the data to ensure that they were compatible with
% the data and accurately represented the comments.

Qualitative analysis of the open-ended questions used thematic
analysis. Thematic analysis has been described as ``a method for
identifying, analysing and reporting patterns (themes) within
data''~\cite{braun+clarke:2006}. This was done by firstly reading
through the qualitative responses and numerically coding the data to
identify whether comments were positive, negative or neutral. The
responses were coded by two researchers to ensure inter-rater
reliability (IRR=0.82). Within these codes potential themes were
identified: ``a theme captures something important about the data in
relation to the research question and represents some level of
patterned response or meaning within the data
set''~\cite[p.82]{braun+clarke:2006}. These themes were reviewed
rigorously against the data to ensure that they were compatible with
the data and accurately represented the comments.


\section{Results}\label{results}

\subsection{Quantitative Data}\label{quantdata}

% Figure~\ref{fig:partagree} shows that those who work within the
% computer science discipline were significantly more likely to say
% that they felt prepared ($\chi^2$(1)= 22.02, p<0,001), confident
% ($\chi^2$(1)= 22.98, p<0,001), supported by their institution
% ($\chi^2$(1)= 4.5, p=0.03), held a good working knowledge of
% appropriate technologies ($\chi^2$(1)= 47.75, p<0,001), had access to
% appropriate technologies ($\chi^2$(1)= 13.19, p<0,001) and were
% confident that their students could access online LT\&A ($\chi^2$(1)=
% 17.16, p<0,001).

%\begin{landscape}
\begin{table*}[]
\resizebox{\textwidth}{!}{%
\begin{tabular}{@{}lrrrr@{}}
\toprule
 &
  \multicolumn{2}{c}{Computer science} &
  \multicolumn{2}{c}{Other disciplines} \\ 
Survey statement &
  \multicolumn{1}{c}{n*} &
  \multicolumn{1}{c}{\%*} &
  \multicolumn{1}{c}{n*} &
  \multicolumn{1}{c}{\%*} \\ \midrule
``\textit{\textit{I feel confident in my ability to facilitate online learning, teaching and assessment}}'' &
  \textbf{215} &
  \textbf{84.0} &
  \textit{1071} &
  \textit{66.5} \\
``\textit{My institution has been supportive in facilitating the move to online learning, teaching and assessment}'' &
  \textbf{220} &
  \textbf{86.6} &
  \textit{1232} &
  \textit{22} \\
``\textit{I have a good working knowledge of the technologies that are available to support learning, teaching and assessment online}'' &
  \textbf{246} &
  \textbf{90.8} &
  \textit{1061} &
  \textit{66.8} \\
``\textit{I can access appropriate technologies to support my online learning, teaching and assessment}'' &
  \textbf{265} &
  \textbf{95.0} &
  \textit{1422} &
  \textit{84.0} \\
``\textit{I am confident that all of my students will be able to access the teaching and assessment that I make available online}'' &
  \textit{122} &
  \textit{50.0} &
  \textbf{538} &
  \textbf{34.3} \\ \bottomrule
\end{tabular}%
}
\caption{Responses to statements by discipline (*number and percent of those agreeing with statement compared to disagreeing)}
\label{tab:respbydisc}
\end{table*}
%\end{landscape}

Table~\ref{tab:respbydisc} shows that those who work in computer
science are significantly more likely to say that they felt prepared
($\chi^2$(1)= 31.47, p$<$0.001), confident ($\chi^2$(1)= 31.44,
p$<$0.001), supported by their institution ($\chi^2$(1)= 9.91, p=0.002),
held a good working knowledge of appropriate technologies
($\chi^2$(1)= 63.66, p$<$0.001), had access to appropriate technologies
($\chi^2$(1)= 23.24, p$<$0.001) and were confident that their students
could access online LT\&A ($\chi^2$(1)= 22.51, p$<$0.001). The figures
presented in {\textbf{bold}} in both Table~\ref{tab:respbydisc} and
Table~\ref{tab:respbysect} had a z score of +1.96, meaning that this
category were significantly more likely than expected to agree with
the statement; the figures in {\emph{italics}} has a z score of -1.96
meaning that this category was significantly less likely than expected
to agree with the statement.

% \begin{figure*}
% \includegraphics[width=0.8\textwidth]{images/particagree.png}
% \caption{Percentage of participants that agree to statements about
%   online LT\&A}
% \label{fig:partagree}
% \end{figure*}

% Table~\ref{tab:binregs} shows the results from binary regression on each
% statement. This demonstrates that the impact of working within the
% computer science discipline remains significant when controlling for
% setting, gender, and years teaching. It also shows that those in
% schools were significantly more likely to agree with the statements
% than those in HE and FE.

The information presented in Table~\ref{tab:respbysect} demonstrates
that within those that responded that they worked within computer
science there was also significant differences between education
sectors. Those who work in schools (84.3\%) were significantly more
likely to say they were prepared than those in HE (69\%) ($\chi^2$(1)=
8.39, p=0.004). Practitioners from schools (91.5\%) stated that they
were significantly more confident than those working in HE (78.7\%)
($\chi^2$(1)= 7.62, p=0.006). Finally, those from schools (39\%) were
significantly less confident that their students would be able access
online LT\&A than those from HE (57.6\%) ($\chi^2$(1)= 8,2,
p=0.004).

% As previously, the figures in {\textbf{bold}} in had a z score of
% +1.96 meaning that this category were significantly more likely than
% expected to agree with the statement; the figures in {\emph{italics}}
% has a z score of -1.96 meaning that this category was significantly
% less likely than expected to agree with the statement.

%\begin{landscape}
\begin{table*}[]
\resizebox{\textwidth}{!}{%
\begin{tabular}{@{}lrrrr@{}}
\toprule
 &
  \multicolumn{2}{c}{School} &
  \multicolumn{2}{c}{HE} \\ 
Survey statement &
  \multicolumn{1}{c}{n*} &
  \multicolumn{1}{c}{\%*} &
  \multicolumn{1}{c}{n*} &
  \multicolumn{1}{c}{\%*} \\\midrule
\textit{"I feel prepared to deliver online learning, teaching and assessment"} &
  \textbf{97} &
  \textbf{84.3} &
  \textit{107} &
  \textit{69.0} \\
"I feel confident in my ability to facilitate online learning, teaching and assessment" &
  \textbf{97} &
  \textbf{91.5} &
  \textit{118} &
  \textit{78.7} \\
\textit{"My institution has been supportive in facilitating the move to online learning, teaching and assessment"} &
  89 &
  85.6 &
  131 &
  87.3 \\
\textit{"I have a good working knowledge of the technologies that are available to support learning, teaching and assessment online"} &
  113 &
  94.2 &
  133 &
  88.1 \\
\textit{"I can access appropriate technologies to support my online learning, teaching and assessment"} &
  111 &
  94.1 &
  154 &
  95.7 \\
\textit{"I am confident that all of my students will be able to access the teaching and assessment that I make available online"} &
  \textit{39} &
  \textit{39.0} &
  \textbf{83} &
  \textbf{57.6} \\ \bottomrule
\end{tabular}%
}
\caption{Responses to statements by sector (*number of percent of those agreeing with statement compared to disagreeing)}
\label{tab:respbysect}
\end{table*}
% \end{landscape}

\subsection{Qualitative Data}\label{qualdata}

\subsubsection{School Practitioners: Positive Aspects}

\begin{quotation}
``{\emph{ICT has gone up massively as a valued skill - hopefully a
    trend that will be reflected and its impact will be increased in
    terms of curriculum timetabling.}}'' [Wales]
\end{quotation}

\begin{quotation}
``{\emph{We are in a pretty unique place because of what we
teach. }}'' [New Zealand]
\end{quotation}

% The qualitative data were coded into positive, negative and neutral
% responses. Of the 102 computer scientists that commented on the impact
% on their role 23 (22.6\%) were positive, 54 (52.9\%) were negative and
% 25 (24.5\%) were neutral.  94 computer scientists provided a comment
% on the impact on their institution, of these 20 (21.3\%) were
% positive, 59 (56.7\%) were negative and 15 (15\%) were
% neutral. Finally, 67 computer scientists commented on the impact on
% their sector, of these 16 (23.9\%) were positive, 36 (53.7\%) were
% negative and 15 (22.4\%) neutral. Key themes were identified within
% the responses. These will now be discussed in relation to computer
% science education.

% \subsubsection{Change as progressive}

Mirroring the quantitative results, the open-ended responses from
school practitioners highlighted the potential benefits that the move
to online LT\&A will have on education. As reflected in the quote
above, respondents acknowledged the positive shift in emphasis on
computer science as a subject: ``{\emph{It may put further emphasis on
    computing as a subject, with so much technology in use}}'' [England].

Respondents also acknowledged the direct impact on their own role as a
subject specialist in computer science: ``{\emph{As ICT coordinator my
role is probably more important now than when in school}}'' [Ireland];
``{\emph{As the resident IT expert, I’m everyone’s new best
friend!}}'' [England]. Thus, reinforcing the practitioner’s own status
in an expert in educational technologies. 

% Computer scientists mentioned a number of progressive and beneficial
% aspects to the change to online LT\&A for the discipline. Most
% prominently, respondents pointed out how the changes have and would
% lead to more recognition of the importance of technology. Common
% responses mentioned the ``{\emph{greater staff awareness of
% educational technologies}}” [school] and that ``{\emph{everyone
% hopefully will now appreciate that digital literacy is important}}''
% [school]. This led to many respondents also recognising how computer
% science as a subject may have its profile raised by the mass move to
% the use of digital technology for learning. One respondent noted
% ``{\emph{it may put further emphasis on computing as a subject, with
% so much technology in use}}'' [school]. As a result, respondents
% foresaw long-term benefits for computer science as a discipline
% within education ``{\emph{ICT has gone up massively as a valued skill
% -- hopefully a trend that will be reflected and its impact will be
% increased in terms of curriculum timetabling}}'' [school].

% \begin{quotation}
% ``{\emph{If used and set up well, it could be amazing.  Breaking down
% barriers to edtech and embracing technology for a connected student
% experience}}'' [school]
% \end{quotation}

% Further opportunities were noted in the advance in educational digital
% infrastructure. A key theme was the acknowledgement of the
% ``{\emph{range, quality and resilience of key digital infrastructure
% and tools}}'' [HE] and how it may ``{\emph{open new opportunities to
% try new online tools}}'' [school]. Furthermore, respondents mentioned
% the potential positive impact of financial investment in digital
% infrastructure. This was coupled with discussion of the opportunities
% for professional development in the area of online LT\&A. It was
% recognised that there had been ``{\emph{more ongoing support for staff
% with technology}}'' [school] and this would lead to long term
% benefits:

Furthermore, respondents spoke positively not only about the impact on
computer science as a discipline, but also about improvements in
cross-curricular digital skills. School practitioners acknowledged the
benefits of all staff upskilling in the area of digital technologies:
``{\emph{Greater staff awareness of education technologies}}'' [New
Zealand]; ``{\emph{As Digital lead for the school it should make
embedding some skills a lot easier as staff have now had a crash
course}}'' [Wales]. More broadly, those in schools also mentioned the
collegiate benefits of the whole school response to the change
``{\emph{This is bringing our staff together in some ways because we
are all collaborating and sharing ideas}}'' [USA].

% \begin{quotation}
% ``{\emph{It will involve a forced skills upskilling in IT skills for a
% number of older members of staff. […] Once over the initial hurdle, I
% feel this could be of benefit long term. But this has only happened
% due to significant support made available to them to support this}}''
% [school]
% \end{quotation}

% There was also discussion about the positive impact this could have on
% pedagogy and practice as it has ``{\emph{opened opportunities for
%     flexible learning}}” [school]. Furthermore, respondents mentioned that
% ``{\emph{it will allow the department to be creative and be innovative
% in the way lessons are presented to students}}'' [school]; therefore,
% indicating potentially innovation online LT\&A methods for computer
% science education.

% There was also acknowledgement that while there may be difficulties in
% terms of equity of access ``{\emph{computer science will be one of the
% least hit as our colleagues and students are among the most capable
% when it comes to operating online}}'' [HE].

% \subsubsection{Change as challenge}

\subsubsection{School Practitioners: Negative Aspects}

\begin{quotation}
``{\emph{As a Computing teacher, most of my resources are already
online. However, teaching programming techniques and complex concepts
of computer science online is difficult.}}''  [UK]
\end{quotation}

However, while a number of positive messages came through, school
computer science practitioners also raised a number of concerns about
the impact of online LT\&A. Reflecting the quantitative results, those
in schools expressed concerns about students’ access to the LT\&A that
would be made available online ``{\emph{I will need to be more
proactive in trying to reach students who may be struggling to
cope. Some are homeless, some don't have laptops, some have less than
ideal home situations}}'' [Ireland]. This particular concern was
raised in response to the resource needed to study computer science
``{\emph{[…] ensuring computing students have access to the internet
and computers […] and have all the essential software downloaded}}''
[Wales].

Furthermore, there was discussion about the appropriate pedagogy for
teaching computer science ``{\emph{For this year, since most of what
needed to be covered has been covered for my AP classes, and since the
AP Exams have been watered down a bit for this year, things are
okay. But if I were to have to go completely to online teaching, I
might as well retire. There's something lost if I can't interact with
my students in a classroom setting. Assessment takes a real dive.}}''
[USA], this lack of face-to-face interaction was a key concern for
school practitioners ``{\emph{now that there is no face-to-face
contact with students, the computer time is very demoralizing}}''
[USA].

More general concerns were also raised about the impact on staff
workload ``{\emph{I will be expected to provide additional content to
support home learning}}'' [New Zealand]. Within this theme, while some
practitioners acknowledged the benefits to the status of their role as
an expert in education technology, others raised concerns about the
impact of this on their workload ``{\emph{So as well as being a
teacher, I have set up the entire school platform, written how to
guides for teachers, made tutorial videos and am doing online training
to all other staff. It's supposed to be my holiday and I have worked
all day every day}}'' [England].

% While acknowledgement of opportunities was clear within the
% qualitative data, respondents also raised concerns about the impact of
% the move to online LT\&A. A key theme within these concerns was
% whether the move to online LT\&A would be as pedagogically beneficial
% to the students as face to face teaching. The quote above summarise
% the concern that top-down messages from institutions imply that `good
% practice is online', however, there is concern that these decisions
% are not pedagogically-driven. A number of respondents raised concerns
% that computer science is ``{\emph{skills-based rather than fact-based.
% I long ago abandoned the traditional lecture as being inappropriate
% for teaching […] .  Too often, moving teaching online means moving
% back to more traditional teaching styles.}}” [HE] and that
% ``{\emph{teaching programming techniques and complex concepts of
% computer science online is difficult}}'' [school]. Furthermore, while
% respondents were aware of the negative impact of the lack of
% face-to-face teaching on teaching computer science, they also
% acknowledged that ``{\emph{social interaction is arguably an important
% component of the student experience}}'' [HE]. Thus, suggesting wider
% pedagogical issues due to the lack of face-to-face teaching.

% Along with a perceived negative impact on effective pedagogy due to
% the move to online LT\&A, concerns were also raised about the equity
% of access to the necessary resources for learning: ``{\emph{online
% learning in CS is heavily dependent on pupils' home access}}''
% [school]. Furthermore, concern was raised about the lack of access to
% labs and computer science specific software for LT\&A: ``{\emph{access
% to specialist laboratories and equipment has been curtailed. Depending
% upon a student’s specialism within computer science their experience
% could be more significantly affected. For example, those studying
% networking or robotics}}'' [HE].

% Also, while respondents acknowledged the lack of necessary physical
% resource as a problem for their students, many acknowledged the lack
% of their own time as a key concern: ``{\emph{the main problem is not
% availability of resource and support, but the time needed to acquire
% skills in using them, for which there is no space in my current
% role}}'' [HE]. The impact of moving resources online on workload was a
% concern raised across the discipline: ``{\emph{huge uncertainty,
% possibly spending all summer converting a large course to online
% without knowing whether/if students will even enrol}}'' [HE]. This
% concern about the fragility of the sector was particularly prominent
% in respondents from HE. This was reinforced by concerns across all
% sectors for the health and wellbeing of both practitioners and
% students, linking back to previous comments regarding equity and
% personal circumstances: ``{\emph{It makes me want to get out of
% teaching computing; the amount of time that I spend at the computer
% feels tiresome at the best of times, and now that there is no
% face-to-face contact with students, the computer time is very
% demoralising}}'' [school] and ``{\emph{My role is shifting towards
% advising and away from teaching. A major challenge will be students'
% mental health, not their ability to write Java code.}}'' [HE].

\subsubsection{HE Practitioners: Positive Aspects}

\begin{quotation}
``{\emph{Overall, this may help us identify techniques that are
particularly helpful in computer science education.}}'' [USA]
\end{quotation}

\begin{quotation}
``{\emph{I think that everybody will begin seeing value in
    technology’s place in education.}}'' [UK]
\end{quotation}

% \begin{quotation}
% ``{\emph{Computer science education is probably a good place to be
% right now.}}'' [USA]
% \end{quotation}

As mirrored within the quantitative data, there were more positive
themes emerging from the school practitioners. However, HE
practitioners also recognised the positive impact of the changes on
the computer science discipline. As demonstrated in the above quotes,
some acknowledged the system learning that may now take place as a
result of the rapid changes; one practitioner stated that
``{\emph{computer science will boom}}'' [Canada].

Furthermore, the potential impact on the wider computer science sector
was also acknowledged ``{\emph{I expect to see an increase in the
number of students in CS education, as CS jobs can typically be
performed remotely (e.g. from home) and are therefore more resilient
in the face of stay at home orders}}'' [UK]. Like with school
practitioners, there was also acknowledgement that those within
computer science are the best equipped to deal with these changes
``{\emph{Computer science will be one of the least hit as our
colleagues and students are among the most capable when it comes to
operating online}}'' [UK].

\begin{quotation}
``{\emph{COVID-19 has been a lightning rod that has catalysed a lot of
    much needed changes in my institution.}}'' [UK]
\end{quotation}

Respondents from HE also mentioned the potential benefits to the HE
sector as a whole: ``{\emph{HE will never be the same. However, this
might provide an opportunity to rethink the role of HE - to educate
rather than to train?}}'' [USA]. One practitioner acknowledged the
benefits on the changes to the structural relationships within their
department: ``{\emph{This disruption has had at least two strong
internal advantages: Everybody has finally made an effort to
transition to online learning; Older faculty have had to rely on the
expertise of younger faculty (whom they were quick to dismiss until
now)}}'' [USA].

Furthermore, there was acknowledgement that the work put in now may
not be wasted in the longer term ``{\emph{Everyone is likely to
leverage the current move online as much as they can. Nobody is going
to waste the work that they are suddenly having to do now}}''
[Canada]. However, as discussed below, there were concerns about the
longer term move to online on the sector.

\subsubsection{HE Practitioners: Negative Aspects}

\begin{quotation}
``{\emph{My role is shifting towards advising and away from
teaching; a major challenge will be students' mental health, not their
ability to write Java code.}}'' [USA]
\end{quotation}

\begin{quotation}
``{\emph{I am concerned that my institution thinks a move online is a
move to more innovative and modern teaching, just by virtue of it
being online.}}'' [England]
\end{quotation}

The key theme emerging from the HE practitioner responses was the
fragility of the sector as a whole. Comments such as ``{\emph{why
would a student choose one school over another when everything is
online?}}'' [Norway]; ``{\emph{I am concerned about how this will
impact recruitment and enrolment next fall}}'' [USA] and ``{\emph{fear
that some Universities may close}}'' [England] summarise practitioners
concerns about the fragility of the HE sector as a whole within this
climate. For computer science in particular, respondents raised
concerns about the retention of staff ``{\emph{Major financial impact
is likely to lead to major staff shortages, particularly in my
discipline, where graduates can all command high salaries in
industry}}'' [USA] along with wider concerns about the potential
impact on the industry: ``{\emph{Produce less qualified graduates due
    to relaxed standards}}'' [USA].

Furthermore, respondents foresaw a longer-term move to online LT\&A
``{\emph{I think there will be greater pressure to do more online
teaching. There will be an attitude that we were successful making
this move in extraordinary circumstances. Surely, we can do the move
permanently}}'' [Philippines] and ``{\emph{I expect we will be asked
to do more online teaching in the future, having now proven it can
work}}'' [Nigeria].

For computer science in particular, there was concern about the access
to specialist software needed for their courses: ``{\emph{Access to
specialist laboratories and equipment has been curtailed. Depending
upon a student’s specialism with Computer Science their experience
could be more significantly affected. For example, those studying
networking or robotics}}'' [South Africa]. In particular concerns were
raised about the more practical aspects of a computer science
programme in HE: ``{\emph{Specifically I work in an area that involves
some hands-on practical projects. These cannot be replicated online,
so the student experience will be significantly changed}}''
[Scotland]. Furthermore, concerns were raised about how effectively certain
aspects of computer science assessment can be done online ``{\emph{the
difficulty in assessing student's knowledge as code is easy enough to
test when doing coursework}}'' [Wales].

\begin{quotation}
``{\emph{Increased workload (already VERY overworked) […]  it is
easier for students to contact me (good) but means the volume of
queries and contacts increases which saps time (bad). Need to do
increased admin […] All in all bad for my career as I can’t do any
science.}}'' [England]
\end{quotation}

As with school practitioners, concern was raised over the impact on
workload. For those in HE there was particular concern about the
impact on other aspects of their academic roles and
responsibilities. In particular, respondents stated that
``{\emph{research will be the hard part}}'' [England]; ``{\emph{this has
massively blown out the proportion of time I expected to spend
teaching, and as such I am not engaging in the research I need to be
doing}}'' [USA].


\section{Discussion}\label{discussion}

% \begin{quotation}
% ``{\emph{Computer science will boom.}}'' [HE, Canada]
% \end{quotation}

\begin{quotation}
``{\emph{This is the beginning of a new era. Things will never be the
same again.}}'' [HE, USA]
\end{quotation}

% % \begin{itemize}
% % \item Computer scientists are more prepared and confident and
% % supported by institution compared to all other disciplines
% % \item Quant this was significant across sector/gender/etc – the impact
% % of being a CS is sig predictor of preparedness when controlling for
% % sector/gender/years teaching, implying CS is sig variable…
% % \item However, while qual data showed some positive responses there
% % were a number of concerns around impact of online LT\&A and their roles going
% % forward
% % \end{itemize}

\subsection{Pedagogy and Practice}

The quantitative data showed that those from computer science were
significantly more positive about the move to online LT\&A in the
immediate aftermath of education closures, then those from other
disciplinary areas. These results are, perhaps, unsurprising, given
the likely proficiency of computer scientists to use
technology. However, they highlight that this confidence with
technology translates to its use for online LT\&A. When this was
broken down further, while those from schools felt more prepared and
confident in their ability to deliver online LT\&A, they were less
confident about their students’ ability to access the material.

% The quantitative data demonstrates that in the immediate aftermath of
% the rapid move to online LT\&A, those working in computer science
% perceived themselves as being significantly more prepared, confident,
% held a good working knowledge of the relevant technologies and felt
% supported by their institution than those in other
% disciplines. Furthermore, they were significantly more likely to agree
% that both themselves and their students could access the online
% LT\&A. While those in secondary schools were significantly more
% likely to agree with these statements than those in FE and HE, being a
% computer scientist remained a significant predictor of these
% viewpoints when controlling for setting, gender, and years working in
% the profession.  These results are, perhaps, unsurprising, given the
% likely proficiency of computer scientists to use technology. However,
% they highlight that this confidence with technology translates to its
% use for online LT\&A.

\begin{quotation}
``{\emph{This is bringing our staff together in some ways because we
are all collaborating and sharing ideas. My principal has been great
about communicating with us on a daily basis.}}'' [school, England]
\end{quotation}

Central to both the positive and negative commentary was high-quality
learning and teaching for computer science, and especially appropriate
pedagogic approaches. While some recognised the potential that moving
teaching online could allow practitioners to be `flexible' and
`creative' with their pedagogy, fostering increased collaboration between teams,
practitioners expressed concern about
how key foundation topics and threshold concepts in computer science
can be taught effectively without face-to-face instruction. Therefore,
while some literature has demonstrated the use of technology to
enhance teaching, a number of practitioners were concerned about its
value and contribution to computer science education, especially for key topics in
computer science, such as programming and mathematical foundations, as
well as more practical or collaborative topics such as robotics and
group software projects.
% Might not want to phrase it like this- but perhaps can add literature
% in the intro which looks at effective use of tech for teaching- and
% refer back to it here.

% \begin{quotation}
% ``{\emph{Specifically I work in an area that involves some hands on practical
% projects. These cannot be replicated online, so the student experience
% will be significantly changed.}}'' [HE]
% \end{quotation}

\subsection{Bridging the Skills Gap}

\begin{quotation}
``{\emph{As a Computing teacher, most of my resources are already
online. However, teaching programming techniques and complex concepts
of computer science online is difficult.}}'' [school, Wales]
\end{quotation}

\begin{quotation}
``{\emph{HE will move increasingly to online provision, sadly. Our
technologies do not currently allow the creation and manipulation of
shared mental representations which is necessary for effective
teaching and learning of mathematics and computer science.}}'' [HE, England]
\end{quotation}

Yet, it could also be argued that the efficiency of online teaching
may be overplayed by institutions. This may be particularly true of
schools, who may be rapidly moving to teaching online, without the
necessary robust digital infrastructure, professional development and
understanding of effective online pedagogy. As noted in the responses,
there may be longer-term positive impact of this technological
upskilling of educational practitioners, however, significant concerns
were raised about the impact on workload due to these changes. There
were also concerns raised about top-down, ``one size fits all'' institutional
approaches, rather than evaluating and addressing
disciplinary-specific challenges and supporting appropriate pedagogic
approaches.

\subsection{Infrastructure}

\begin{quotation}
``{\emph{Delay in critical upgrades to servers and increase in
infrastructure.  Need to expend further funds to have suitable
hardware to loan to staff in these circumstances.}}'' [school, England]
\end{quotation}

Another theme that was acknowledged as a significant challenge was the
demand on educational digital infrastructure. While practitioners
acknowledged the potential opportunities of institutional financial
investment in digital infrastructure, concern was raised about equity
of access to these recourses. While it was acknowledged by some HE
practitioners that computer science students may be the least affected
by this, there was broader concern for those that may not be able to
access appropriate technologies (especially if there was a requirement
for specialist equipment or software), and that it was easy to make
assumptions about how and when students are able to engage with online
learning. This concern was more consistently expressed by school
practitioners.

% \begin{quotation}
% ``{\emph{I teach programming and robotics.  Programming transitions
% easily to online learning but Robotics does not.  It remains to be
% seen if we can continue that class.}}'' [HE]
% \end{quotation}

\begin{quotation}
``{\emph{The difficulty is how to provide alternatives to specialised
laboratory provision. We also have large numbers of international
students, some of whom have now gone back to their home
countries. Some of these have very poor or no access to technology and
keeping in touch with them is challenging. Luckily our sector of
education, computer science, means that both staff and students tend
to have good knowledge of digital technologies and how they can
support online learning but care still needs to be taken as not all
students have good access from home or can adapt easily to an online
version of education.}}'' [HE, UK]
\end{quotation}

\subsection{Limitations}

It is also necessary to identify the limitations of this research and
to highlight the potential for how it can support future research in
this area. As this research was conducted in the immediate aftermath
of the move to online LT\&A it could be argued that, due to the rapid
changes in the situation since March 2020, attitudes may have changed
since this data was collected. Furthermore, this study has grouped
together international computer science practitioners from across
various educational settings. It could be argued that the difference
in experience of these practitioners is vast and, consequently, it is
difficult to recognise them as a homogenous group. However, the
coherence from the quantitative and qualitative results offers some
strength to the insights into international computer science education
practitioners' perceptions during these radical changes. The results
highlight the longer-term opportunities and challenges that the move
may bring about. Furthermore, follow-up research should be conducted
in order to better understand how perceptions have changed since this
data was collected.
% mention that using UK JACS codes wasn't perfect?

% Firstly, this study has grouped together international computer
% science practitioners from across various educational settings. It
% could be argued that the difference in experience of these
% practitioners is vast and, consequently, it is difficult to recognise
% them as a homogenous group. However, logistic regression analysis
% demonstrated that even while controlling for setting, identifying as a
% computer scientist was a significant predictor of responses to the key
% statements. Across all participants, secondary school practitioners
% were significantly more positive than those in HE and FE. Therefore,
% future research should investigate why this trend has been identified,
% and whether those in HE and FE can learn from practitioners in schools
% about effective methods of online LT\&A.

\section{Conclusions and Looking Ahead}\label{conclusions}

% % Many of the issues expressed in the survey are generalisable and
% % common across their respective sectors; jobs, funding, sustainability
% % of model, admissions, assessment, robustness of quals and examination
% % system
Many of the challenges and opportunities presented by COVID-19 and the
rapid shift to ``emergency remote teaching'' as identified in this
survey could be applied more broadly across the various international
educational settings. In particular, there are significant concerns
regarding impact on job precarity and security, career progression,
financial sustainability of
institutions~\cite{watermeyer-et-al:he2020}, robustness of the
qualifications and examinations system, issues of equity and access to
technology, as well as the health and wellbeing of practitioners and
students due to increased workloads and expectations. However, it was
clear that school practitioners were frequently more positive due to
the perceived rise in status of their ``key worker'' role and for
their discipline more generally. It is also important to acknowledge
the ongoing media narrative regarding online teaching being perceived
as lower quality than face-to-face teaching (especially for
HE)~\cite{paechter+maier:ihe2010,scbbcnews:2020}; however, teaching
quality is more important than how lessons are
delivered~\cite{eefremote:2020}, while technology can be used to
improve the quality of explanations and modelling, and can play a role
in improving assessment and feedback~\cite{eefdigtech:2019}.

% \footnote{Also see: ``{\emph{Academic lives are in transition}}''
% \url{https://wonkhe.com/blogs/academic-lives-are-in-transition/} (May
% 2020) and ``{\emph{Forced shift to online teaching in coronavirus
% pandemic unleashes educators’ deepest job fears}}''
% \url{https://www.natureindex.com/news-blog/forced-shift-to-online-teaching-in-coronavirus-pandemic-unleashes-educators-deepest-job-fears-}
% (April 2020)}

The rapid adoption of digital technologies for almost all activities
that could previously have taken place within the physical space of an
educational institution presents opportunities to rethink how many
academic practices might take place in virtual environments. These
resultant shifts in culture, identity, and new demands on educational
leadership and management -- especially in
schools~\cite{slameditorial:2020}) -- and perhaps specific challenges
for computer science as a discipline. However, reshaping the
post-pandemic digital structure of education also risks exacerbating
existing inequalities in the use of digital technologies (especially
in the context of digital exclusion and digital/data
poverty~\cite{watts:2020,beaunoyer-et-al:2020}), as well as opening up
new areas of academic life to surveillance and
control~\cite{carriganlseblog:2020}, directly linking to wider
priorities surrounding the importance of legal, social, ethical and
professional issues in computer science education.

% % Looking ahead for the discipline
% % Positive for overall use/impact/perceptions of technology use in
% % education, with knock-on effects for the discipline of computer
% % science, pedagogy and its prominence in schools, quals, etc.
% crick+sentance:2011,wgictreview:2013,brown-et-al-toce2014,moller+crick:jce2018
However, there are a number of specific issues for computer science
practitioners that provides valuable insight and context for the
discipline as we move with some uncertainty towards the next academic
year and beyond. In particular, the increased prominence of technology
in an educational context provides opportunities for showcasing the
importance of cross-curricular digital and data skills, as well as the
explicit value of computer science as a STEM academic discipline. This
clearly resonates with recent international computer science curricula
and qualifications reforms, especially as computer science is starting
to become increasingly established as a school-level
subject~\cite{bell:2014,brown-et-al-toce2014,gal-ezer+stephenson:2014,raman-et-al:2015}.
There is also an increasing focus on identifying and refining
effective pedagogic approaches for LT\&A on key foundation topics in
computer science -- and especially for CS1 -- such as mathematical
foundations, programming and
cybersecurity~\cite{davenport-et-al:latice2016,murphy-et-al:programming2017,simon-et-al:sigcse2018,crick-et-al:fie2019,davenport+crick:fmfun2019,prickett-et-al:iticse2020,crick-et-al:fie2020}. However,
there are concerns of top-down, ``one-size-fits-all'' institutional or
national approaches that do not recognise the unique characteristics
of LT\&A in computer science across the various settings and
levels. Further work is required to better identify, evaluate and
share best practice for some of these areas, especially with regards
to assessment, certification and qualifications.

% We have also seen the
% announcement of a review of academic accreditation of computer science
% degrees~\cite{crick-et-al:fie2019,crick-et-al-accred:cep2020} launched
% at the start of July, to ``ensure that computing graduates have the
% skills needed to drive economic recovery as data science and AI change
% the industry post
% COVID-19''\footnote{See:~\url{https://www.bcs.org/more/about-us/press-office/press-releases/review-of-academic-accreditation-of-computer-science-degrees-launched/}}.
%~\cite{bcscsdegrev:2020}.

% % \cite{bsasciecareers:2020}
% \footnote{See:~\url{https://www.britishscienceassociation.org/blog/young-people-are-more-interested-in-a-scientific-career-as-a-result-of-covid-19}}
Finally, it is clear there will be a huge demand for digital skills
and
infrastructure~\cite{tryfonas+crick:petra2018,davenport-et-al:educon2020,baker:2020}
to support the global post-COVID economic
renewal~\cite{nadellaft:2020}. Recent evidence from the UK suggests
that young people are more interested in science and technology
careers as a result of COVID-19~\cite{bsasciecareers:2020}, alongside
opportunities to promote cross-curricula and interdisciplinary
approaches in school STEM lessons when addressing wider societal
issues~\cite{reiss:2020}. Based on the data obtained from this rapid
response survey of international computer science practitioners, we
anticipate further evaluation and development of best practice for
online LT\&A for computer science as we move into the 2020-2021
academic year and beyond. Furthermore, it is imperative that follow-up
studies are conducted to capture the longer-term impact to computer
science education, especially as it appears that the virus may have to
be tolerated on an indefinite basis~\cite{kissler-et-al:2020}.

% \section*{Acknowledgements}

% This work was supported by the {\emph{Institute of Coding}} (IoC),
% which received \pounds20m of funding from the Office for Students
% (OfS), as well as support from the Higher Education Funding Council
% for Wales (HEFCW).

% We are grateful to the IoC Observatory staff, especially Dr Fiona
% MacGill (University of Bath), for the data collection. The first
% author is also grateful to Dr Matt Dickson (University of Bath) for
% discussions about undergraduate degrees and labour market returns in
% the UK~\cite{DfE2018d}, but any mistakes are the authors' alone.

% trigger a \newpage just before the given reference
% number - used to balance the columns on the last page
% adjust value as needed - may need to be readjusted if
% the document is modified later
%\IEEEtriggeratref{50}
% The "triggered" command can be changed if desired:
%\IEEEtriggercmd{\enlargethispage{-5in}}

\bibliographystyle{IEEEtran}      % basic style, author-year citations
\bibliography{EDUCON2021} 

\end{document}
